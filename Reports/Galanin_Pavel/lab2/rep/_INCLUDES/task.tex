\begin{center}
    \textbf{\titlePageWorkType~№\titlePageWorkNumber~\titlePageWorkPart}
\end{center}

\textbf{Тема}: <<\titlePageTopic>>

%\textbf{Цель работы}: 

\begin{center}
    \textbf{Ход работы}:
\end{center}

Написать ассемблерную вставку, реализующую следующую обработку строки: согласно варианту.
Оформить ее в виде отдельной функции.
Реализовать данную обработку строки также в виде функции на С++.
Сравнить быстродействие обоих вариантов.
В отчете отразить выводы.
Для разработки использовать MS Visual Studio.

\textbf{Варианты}:
\begin{enumerate}
    \item[1.] Перевернуть строку.
    \item[2.] Поменять четные символы с нечетными.
    \item[3.] Даны 4 строки. Поменять 1-ю с 3-ей, 2-ю с 4-й.
    \item[4.] Даны 2 строки. Совместить четные символы одной строки с нечентными другой.
    \item[5.] Даны 2 строки. Совместить половину строки 1 с половиной строки 2.
    \item[6.] Нечетные символы заменить на +.
    \item[7.] Совместить 2-е строки. Совпадающие символы заменить на 0.
    \item[8.] Сместить все символы на 1-н вперед циклично.
    \item[9.] Перевернуть две половины строки.
    \item[10.] Сместить все символы на 1-н назад циклично.
    \item[11.] Заменить пробелы на символ табуляции.
    \item[12.] Заменить пробелы на символ табуляции.
    \item[13.] Удалить повторяющиеся пробелы, также пробелы в начале и в конце строки.
\end{enumerate}

Пример функции с ассемблерной вставкой (С++ Visual Studio): 

\begin{lstlisting}[
    name=Пример assembler'ной вставки,
    basicstyle=\ttfamily\scriptsize,
]
    int f()
    {
        __asm
        {
            mov eax, 1
            int 3
        }
        return 0;
    }
\end{lstlisting}

\newpage

\begin{center}
    \textbf{Вариант 3}
\end{center}

\textbf{Условие}:
Даны 4 строки. Поменять 1-ю с 3-ей, 2-ю с 4-й.

\textbf{Решение}:

\lstinputlisting[
    language=C++,
    basicstyle=\ttfamily\footnotesize,
]
{../../src/option3/option3/App/main.cpp}
