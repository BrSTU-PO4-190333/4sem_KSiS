\subparagraph{Задание 4.5}

\textbf{Условие}:
Восстановить исходную программу и создать загрузочный модуль Lab2.exe (при этом использовать команду \textbf{d:\textbackslash\/tasm\textbackslash\/tasm /zi lab2.asm} на этапе трансляции и команду lab2.obj \textbf{d:\textbackslash\/tasm\textbackslash\/tlink /v lab2.obj} на этапе компоновки. Привести в отчете назначение ключей \textbf{/zi} и \textbf{/v}).

\textbf{Решение}:

\begin{lstlisting}[language=Terminal]
$ d:\tasm\tasm /zi lab2.asm
\end{lstlisting}

\begin{lstlisting}[language=Out]
Turbo Assembler  Version 2.0  Copyright (c) 1988, 1990 Borland International

Assembling file:   lab2.asm
Error messages:    None
Warning messages:  None
Passes:            1
Remaining memory:  473k
\end{lstlisting}

Размер объектного файла LAB2.OBJ был \textbf{309 байт}, стал \textbf{577 байт} (в 1.867 раз больше).

\begin{lstlisting}[language=Terminal]
$ d:\tasm\tlink /v lab2.obj
\end{lstlisting}

\begin{lstlisting}[language=Out]
Turbo Link  Version 2.0  Copyright (c) 1987, 1989 Borland International
\end{lstlisting}

Размер исполняемого файла LAB2.EXE был \textbf{597 байт}, стал \textbf{1,32 КБ (1 360 байт)} (в 2.278 раз больше).

Ключ \textbf{/zi} управляет включением в объектный файл номеров строк исходной программы и другой информации, не требуемой при выполнении программы, но используемой отладчиком.

Ключ \textbf{/v} передает в загрузочный файл символьную информацию, позволяющую отладчику TD выводить на экран полный текст исходной программы, включая метки, комментарии и прочее.
