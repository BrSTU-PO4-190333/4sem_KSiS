\begin{center}
   \textbf{\titlePageWorkType~№\titlePageWorkNumber~\titlePageWorkPart}
\end{center}

\textbf{Тема}: <<\titlePageTopic>>

\textbf{Цель работы}: Знакомство со структорой ассемблерной программы, процессом её получения и основными возможностями программ TASM, TLINK.

\begin{center}
   \textbf{Ход работы}:
\end{center}



\subparagraph{Задание 2.1} 

\textbf{Условие}: Пользуясь правилами оформления ассемблерных программ, набрать программу Hello в файл LAB01.ASM. Получить выполняемый файл и запустить его. Каков результат?

\textbf{Решение}: 

\lstinputlisting[
   name=LAB01.ASM,
   language={[x86masm]Assembler},
   tabsize=8,
   basicstyle=\ttfamily\scriptsize,
]
{../../src/LAB01.ASM}

\newpage



\begin{lstlisting}[language=Terminal]
$ C:\tasm\tasm.exe LAB01.ASM
\end{lstlisting}

\begin{lstlisting}[language=Out]
Turbo Assembler  Version 2.0  Copyright (c) 1988, 1990 Borland International

Assembling file:   LAB01.ASM
Error messages:    None
Warning messages:  None
Passes:            1
Remaining memory:  475k
\end{lstlisting}

Создался файл LAB01.OBJ.



\begin{lstlisting}[language=Terminal]
$ C:\tasm\tlink.exe LAB01.OBJ
\end{lstlisting}

\begin{lstlisting}[language=Out]
Turbo Link  Version 2.0  Copyright (c) 1987, 1989 Borland International
\end{lstlisting}

Создались файлы: LAB01.EXE, LAB01.MAP.



\begin{lstlisting}[language=Terminal]
$ LAB01.EXE
\end{lstlisting}

\begin{lstlisting}[language=Out]
Hello.
\end{lstlisting}



\subparagraph{Задание 2.2}

\textbf{Условие}: Определить, какие файлы создались при получении выполняемого файла. Пользуясь редактором, определить внутренний формат (текстовый или бинарный) КАЖДОГО из них.

\textbf{Решение}: 

\begin{table}[!ht]
   \begin{center}
      \begin{tabular}{|c|c|c|c|}
         \hline
         после команды  & файл      & тип             & что это               \\ \hline
         \hline
         tasm           & LAB01.OBJ & binary format   & объектный файл        \\ \hline
         tlink          & LAB01.EXE & binary format   & исполняемый файл      \\ \hline
         tlink          & LAB01.MAP & text format     & файл карта загрузки   \\ \hline
      \end{tabular}
   \end{center}
\end{table}



\subparagraph{Задание 2.3}

\textbf{Условие}: Получить список ключей TASM, задав пустую командную строку.

\textbf{Решение}: 

\begin{lstlisting}[language=Terminal]
$ C:\tasm\tasm.exe
\end{lstlisting}

\begin{lstlisting}[language=Out]
Turbo Assembler  Version 2.0  Copyright (c) 1988, 1990 Borland International
Syntax:  TASM [options] source [,object] [,listing] [,xref]
/a,/s         Alphabetic or Source-code segment ordering
/c            Generate cross-reference in listing
/dSYM[=VAL]   Define symbol SYM = 0, or = value VAL
/e,/r         Emulated or Real floating-point instructions
/h,/?         Display this help screen
/iPATH        Search PATH for include files
/jCMD         Jam in an assembler directive CMD (eg. /jIDEAL)
/kh#,/ks#     Hash table capacity #, String space capacity #
/l,/la        Generate listing: l=normal listing, la=expanded listing
/ml,/mx,/mu   Case sensitivity on symbols: ml=all, mx=globals, mu=none
/mv#          Set maximum valid length for symbols
/m#           Allow # multiple passes to resolve forward references
/n            Suppress symbol tables in listing
/o,/op        Generate overlay object code, Phar Lap-style 32-bit fixups
/p            Check for code segment overrides in protected mode
/q            Suppress OBJ records not needed for linking
/t            Suppress messages if successful assembly
/w0,/w1,/w2   Set warning level: w0=none, w1=w2=warnings on
/w-xxx,/w+xxx Disable (-) or enable (+) warning xxx
/x            Include false conditionals in listing
/z            Display source line with error message
/zi,/zd       Debug info: zi=full, zd=line numbers only
\end{lstlisting}



\subparagraph{Задание 2.4}

\textbf{Условие}: Получить файл AAA.OBJ из файла LAB01.ASM. Обратить внимание на сообщения TASM. Что изменилось? Выполнить трансляцию ещё раз с ключом /t. На что он влияет?

\textbf{Решение}:

\begin{lstlisting}[language=Terminal]
$ C:\tasm\tasm.exe LAB01.ASM AAA.OBJ
\end{lstlisting}
   
Создался объектный файл AAA.OBJ (прошлый раз был LAB01.OBJ).

\begin{lstlisting}[language=Out]
Turbo Assembler  Version 2.0  Copyright (c) 1988, 1990 Borland International

Assembling file:   LAB01.ASM  to  AAA.OBJ
Error messages:    None
Warning messages:  None
Passes:            1
Remaining memory:  475k
\end{lstlisting}



\subparagraph{Задание 2.5}

\textbf{Условие}: Получить из файла AAA.OBJ файл BBB.EXE без генерации карты  загрузки. Догадайтесь, как это сделать, по аналогии с TASM. Каков вид командной строки?

\textbf{Решение}: 

\begin{lstlisting}[language=Terminal]
$ C:\tasm\tlink.exe
\end{lstlisting}

\begin{lstlisting}[language=Out]
Turbo Link  Version 2.0  Copyright (c) 1987, 1989 Borland International
Syntax:  TLINK objfiles, exefile, mapfile, libfiles
@xxxx indicates use response file xxxx
Options: /m = map file with publics
         /x = no map file at all
         /i = initialize all segments
         /l = include source line numbers
         /s = detailed map of segments
         /n = no default libraries
         /d = warn if duplicate symbols in libraries
         /c = lower case significant in symbols
         /3 = enable 32-bit processing
         /v = include full symbolic debug information
         /e = ignore Extended Dictionary
         /t = create COM file
\end{lstlisting}

\begin{lstlisting}[language=Terminal]
$ C:\tasm\tlink.exe AAA.OBJ, BBB.EXE /x
\end{lstlisting}
   
\begin{lstlisting}[language=Out]
Turbo Link  Version 2.0  Copyright (c) 1987, 1989 Borland International
\end{lstlisting}

Создался только исполняемый файл BBB.EXE (раньше были файлы LAB01.EXE - исполняемый, и LAB01.MAP - карта загрузки).
